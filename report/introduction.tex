Dans le cadre du module de Compilation, nous avons à réaliser un projet individuel ou en binôme.

\section{Objectifs}

Le but de ce projet est de créer un compilateur prenant un fichier écrit en FreeBASIC et générant un fichier C.

Il est impossible dans le temps imparti de créer un compilateur exhaustif, le but de ce module était d'arriver à implémenter le plus de règles possible pour le compilateur.


\section{Contraintes}

\begin{itemize}

\item Langage de programmation: OCaml
\item Outils: OCamlLex et OcamlYacc
\item Langage source: FreeBASIC
\item Langage destination: C
\item Le plus de règles possibles
\item Compilable sous linux

\end{itemize}

\section{Composition d'un compilateur}
Un compilateur est composé de deux principales parties: Un analyseur lexical et un analyseur syntaxique. L'analyseur lexical permet de lire le flux d'entrée et de spécifier le lexique grâce à des expressions régulières. 

Si un "mot" ne correspont à aucune des expressions régulières, une erreur lexicale est renvoyée. Une fois qu'un mot du lexique est trouvé, le token correspondant est passé à l'analyseur syntaxique. L'analyseur syntaxique lui va analyser l'ordre dans lequel les tokens vont lui être passées. Tant que l'ordre correpond aux rêgles syntaxiques du langage source, la traduction vers le langage cible est effectué, sinon une erreur de syntaxe est renvoyée. 

\section{Règles implémentées et ordre d'implémentation}

Les différentes règles ont été implémentées une par une, voici l'ordre dans lequel est l'ont été:
\begin{itemize}
    \item EOF
    \item EOL
    \item PRINT
    \item DIM AS STRING INTEGER
    \item ERROR
    \item IF THEN ELSE
    \item FOR WHILE DO LOOP NEXT
    \item OPERATOR
    \item OPERATION
    \item CONST
\end{itemize}


